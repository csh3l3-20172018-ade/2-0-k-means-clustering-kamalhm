
% Default to the notebook output style

    


% Inherit from the specified cell style.




    
\documentclass[11pt]{article}

    
    
    \usepackage[T1]{fontenc}
    % Nicer default font (+ math font) than Computer Modern for most use cases
    \usepackage{mathpazo}

    % Basic figure setup, for now with no caption control since it's done
    % automatically by Pandoc (which extracts ![](path) syntax from Markdown).
    \usepackage{graphicx}
    % We will generate all images so they have a width \maxwidth. This means
    % that they will get their normal width if they fit onto the page, but
    % are scaled down if they would overflow the margins.
    \makeatletter
    \def\maxwidth{\ifdim\Gin@nat@width>\linewidth\linewidth
    \else\Gin@nat@width\fi}
    \makeatother
    \let\Oldincludegraphics\includegraphics
    % Set max figure width to be 80% of text width, for now hardcoded.
    \renewcommand{\includegraphics}[1]{\Oldincludegraphics[width=.8\maxwidth]{#1}}
    % Ensure that by default, figures have no caption (until we provide a
    % proper Figure object with a Caption API and a way to capture that
    % in the conversion process - todo).
    \usepackage{caption}
    \DeclareCaptionLabelFormat{nolabel}{}
    \captionsetup{labelformat=nolabel}

    \usepackage{adjustbox} % Used to constrain images to a maximum size 
    \usepackage{xcolor} % Allow colors to be defined
    \usepackage{enumerate} % Needed for markdown enumerations to work
    \usepackage{geometry} % Used to adjust the document margins
    \usepackage{amsmath} % Equations
    \usepackage{amssymb} % Equations
    \usepackage{textcomp} % defines textquotesingle
    % Hack from http://tex.stackexchange.com/a/47451/13684:
    \AtBeginDocument{%
        \def\PYZsq{\textquotesingle}% Upright quotes in Pygmentized code
    }
    \usepackage{upquote} % Upright quotes for verbatim code
    \usepackage{eurosym} % defines \euro
    \usepackage[mathletters]{ucs} % Extended unicode (utf-8) support
    \usepackage[utf8x]{inputenc} % Allow utf-8 characters in the tex document
    \usepackage{fancyvrb} % verbatim replacement that allows latex
    \usepackage{grffile} % extends the file name processing of package graphics 
                         % to support a larger range 
    % The hyperref package gives us a pdf with properly built
    % internal navigation ('pdf bookmarks' for the table of contents,
    % internal cross-reference links, web links for URLs, etc.)
    \usepackage{hyperref}
    \usepackage{longtable} % longtable support required by pandoc >1.10
    \usepackage{booktabs}  % table support for pandoc > 1.12.2
    \usepackage[inline]{enumitem} % IRkernel/repr support (it uses the enumerate* environment)
    \usepackage[normalem]{ulem} % ulem is needed to support strikethroughs (\sout)
                                % normalem makes italics be italics, not underlines
    

    
    
    % Colors for the hyperref package
    \definecolor{urlcolor}{rgb}{0,.145,.698}
    \definecolor{linkcolor}{rgb}{.71,0.21,0.01}
    \definecolor{citecolor}{rgb}{.12,.54,.11}

    % ANSI colors
    \definecolor{ansi-black}{HTML}{3E424D}
    \definecolor{ansi-black-intense}{HTML}{282C36}
    \definecolor{ansi-red}{HTML}{E75C58}
    \definecolor{ansi-red-intense}{HTML}{B22B31}
    \definecolor{ansi-green}{HTML}{00A250}
    \definecolor{ansi-green-intense}{HTML}{007427}
    \definecolor{ansi-yellow}{HTML}{DDB62B}
    \definecolor{ansi-yellow-intense}{HTML}{B27D12}
    \definecolor{ansi-blue}{HTML}{208FFB}
    \definecolor{ansi-blue-intense}{HTML}{0065CA}
    \definecolor{ansi-magenta}{HTML}{D160C4}
    \definecolor{ansi-magenta-intense}{HTML}{A03196}
    \definecolor{ansi-cyan}{HTML}{60C6C8}
    \definecolor{ansi-cyan-intense}{HTML}{258F8F}
    \definecolor{ansi-white}{HTML}{C5C1B4}
    \definecolor{ansi-white-intense}{HTML}{A1A6B2}

    % commands and environments needed by pandoc snippets
    % extracted from the output of `pandoc -s`
    \providecommand{\tightlist}{%
      \setlength{\itemsep}{0pt}\setlength{\parskip}{0pt}}
    \DefineVerbatimEnvironment{Highlighting}{Verbatim}{commandchars=\\\{\}}
    % Add ',fontsize=\small' for more characters per line
    \newenvironment{Shaded}{}{}
    \newcommand{\KeywordTok}[1]{\textcolor[rgb]{0.00,0.44,0.13}{\textbf{{#1}}}}
    \newcommand{\DataTypeTok}[1]{\textcolor[rgb]{0.56,0.13,0.00}{{#1}}}
    \newcommand{\DecValTok}[1]{\textcolor[rgb]{0.25,0.63,0.44}{{#1}}}
    \newcommand{\BaseNTok}[1]{\textcolor[rgb]{0.25,0.63,0.44}{{#1}}}
    \newcommand{\FloatTok}[1]{\textcolor[rgb]{0.25,0.63,0.44}{{#1}}}
    \newcommand{\CharTok}[1]{\textcolor[rgb]{0.25,0.44,0.63}{{#1}}}
    \newcommand{\StringTok}[1]{\textcolor[rgb]{0.25,0.44,0.63}{{#1}}}
    \newcommand{\CommentTok}[1]{\textcolor[rgb]{0.38,0.63,0.69}{\textit{{#1}}}}
    \newcommand{\OtherTok}[1]{\textcolor[rgb]{0.00,0.44,0.13}{{#1}}}
    \newcommand{\AlertTok}[1]{\textcolor[rgb]{1.00,0.00,0.00}{\textbf{{#1}}}}
    \newcommand{\FunctionTok}[1]{\textcolor[rgb]{0.02,0.16,0.49}{{#1}}}
    \newcommand{\RegionMarkerTok}[1]{{#1}}
    \newcommand{\ErrorTok}[1]{\textcolor[rgb]{1.00,0.00,0.00}{\textbf{{#1}}}}
    \newcommand{\NormalTok}[1]{{#1}}
    
    % Additional commands for more recent versions of Pandoc
    \newcommand{\ConstantTok}[1]{\textcolor[rgb]{0.53,0.00,0.00}{{#1}}}
    \newcommand{\SpecialCharTok}[1]{\textcolor[rgb]{0.25,0.44,0.63}{{#1}}}
    \newcommand{\VerbatimStringTok}[1]{\textcolor[rgb]{0.25,0.44,0.63}{{#1}}}
    \newcommand{\SpecialStringTok}[1]{\textcolor[rgb]{0.73,0.40,0.53}{{#1}}}
    \newcommand{\ImportTok}[1]{{#1}}
    \newcommand{\DocumentationTok}[1]{\textcolor[rgb]{0.73,0.13,0.13}{\textit{{#1}}}}
    \newcommand{\AnnotationTok}[1]{\textcolor[rgb]{0.38,0.63,0.69}{\textbf{\textit{{#1}}}}}
    \newcommand{\CommentVarTok}[1]{\textcolor[rgb]{0.38,0.63,0.69}{\textbf{\textit{{#1}}}}}
    \newcommand{\VariableTok}[1]{\textcolor[rgb]{0.10,0.09,0.49}{{#1}}}
    \newcommand{\ControlFlowTok}[1]{\textcolor[rgb]{0.00,0.44,0.13}{\textbf{{#1}}}}
    \newcommand{\OperatorTok}[1]{\textcolor[rgb]{0.40,0.40,0.40}{{#1}}}
    \newcommand{\BuiltInTok}[1]{{#1}}
    \newcommand{\ExtensionTok}[1]{{#1}}
    \newcommand{\PreprocessorTok}[1]{\textcolor[rgb]{0.74,0.48,0.00}{{#1}}}
    \newcommand{\AttributeTok}[1]{\textcolor[rgb]{0.49,0.56,0.16}{{#1}}}
    \newcommand{\InformationTok}[1]{\textcolor[rgb]{0.38,0.63,0.69}{\textbf{\textit{{#1}}}}}
    \newcommand{\WarningTok}[1]{\textcolor[rgb]{0.38,0.63,0.69}{\textbf{\textit{{#1}}}}}
    
    
    % Define a nice break command that doesn't care if a line doesn't already
    % exist.
    \def\br{\hspace*{\fill} \\* }
    % Math Jax compatability definitions
    \def\gt{>}
    \def\lt{<}
    % Document parameters
    \title{KMeansClustering}
    
    
    

    % Pygments definitions
    
\makeatletter
\def\PY@reset{\let\PY@it=\relax \let\PY@bf=\relax%
    \let\PY@ul=\relax \let\PY@tc=\relax%
    \let\PY@bc=\relax \let\PY@ff=\relax}
\def\PY@tok#1{\csname PY@tok@#1\endcsname}
\def\PY@toks#1+{\ifx\relax#1\empty\else%
    \PY@tok{#1}\expandafter\PY@toks\fi}
\def\PY@do#1{\PY@bc{\PY@tc{\PY@ul{%
    \PY@it{\PY@bf{\PY@ff{#1}}}}}}}
\def\PY#1#2{\PY@reset\PY@toks#1+\relax+\PY@do{#2}}

\expandafter\def\csname PY@tok@w\endcsname{\def\PY@tc##1{\textcolor[rgb]{0.73,0.73,0.73}{##1}}}
\expandafter\def\csname PY@tok@c\endcsname{\let\PY@it=\textit\def\PY@tc##1{\textcolor[rgb]{0.25,0.50,0.50}{##1}}}
\expandafter\def\csname PY@tok@cp\endcsname{\def\PY@tc##1{\textcolor[rgb]{0.74,0.48,0.00}{##1}}}
\expandafter\def\csname PY@tok@k\endcsname{\let\PY@bf=\textbf\def\PY@tc##1{\textcolor[rgb]{0.00,0.50,0.00}{##1}}}
\expandafter\def\csname PY@tok@kp\endcsname{\def\PY@tc##1{\textcolor[rgb]{0.00,0.50,0.00}{##1}}}
\expandafter\def\csname PY@tok@kt\endcsname{\def\PY@tc##1{\textcolor[rgb]{0.69,0.00,0.25}{##1}}}
\expandafter\def\csname PY@tok@o\endcsname{\def\PY@tc##1{\textcolor[rgb]{0.40,0.40,0.40}{##1}}}
\expandafter\def\csname PY@tok@ow\endcsname{\let\PY@bf=\textbf\def\PY@tc##1{\textcolor[rgb]{0.67,0.13,1.00}{##1}}}
\expandafter\def\csname PY@tok@nb\endcsname{\def\PY@tc##1{\textcolor[rgb]{0.00,0.50,0.00}{##1}}}
\expandafter\def\csname PY@tok@nf\endcsname{\def\PY@tc##1{\textcolor[rgb]{0.00,0.00,1.00}{##1}}}
\expandafter\def\csname PY@tok@nc\endcsname{\let\PY@bf=\textbf\def\PY@tc##1{\textcolor[rgb]{0.00,0.00,1.00}{##1}}}
\expandafter\def\csname PY@tok@nn\endcsname{\let\PY@bf=\textbf\def\PY@tc##1{\textcolor[rgb]{0.00,0.00,1.00}{##1}}}
\expandafter\def\csname PY@tok@ne\endcsname{\let\PY@bf=\textbf\def\PY@tc##1{\textcolor[rgb]{0.82,0.25,0.23}{##1}}}
\expandafter\def\csname PY@tok@nv\endcsname{\def\PY@tc##1{\textcolor[rgb]{0.10,0.09,0.49}{##1}}}
\expandafter\def\csname PY@tok@no\endcsname{\def\PY@tc##1{\textcolor[rgb]{0.53,0.00,0.00}{##1}}}
\expandafter\def\csname PY@tok@nl\endcsname{\def\PY@tc##1{\textcolor[rgb]{0.63,0.63,0.00}{##1}}}
\expandafter\def\csname PY@tok@ni\endcsname{\let\PY@bf=\textbf\def\PY@tc##1{\textcolor[rgb]{0.60,0.60,0.60}{##1}}}
\expandafter\def\csname PY@tok@na\endcsname{\def\PY@tc##1{\textcolor[rgb]{0.49,0.56,0.16}{##1}}}
\expandafter\def\csname PY@tok@nt\endcsname{\let\PY@bf=\textbf\def\PY@tc##1{\textcolor[rgb]{0.00,0.50,0.00}{##1}}}
\expandafter\def\csname PY@tok@nd\endcsname{\def\PY@tc##1{\textcolor[rgb]{0.67,0.13,1.00}{##1}}}
\expandafter\def\csname PY@tok@s\endcsname{\def\PY@tc##1{\textcolor[rgb]{0.73,0.13,0.13}{##1}}}
\expandafter\def\csname PY@tok@sd\endcsname{\let\PY@it=\textit\def\PY@tc##1{\textcolor[rgb]{0.73,0.13,0.13}{##1}}}
\expandafter\def\csname PY@tok@si\endcsname{\let\PY@bf=\textbf\def\PY@tc##1{\textcolor[rgb]{0.73,0.40,0.53}{##1}}}
\expandafter\def\csname PY@tok@se\endcsname{\let\PY@bf=\textbf\def\PY@tc##1{\textcolor[rgb]{0.73,0.40,0.13}{##1}}}
\expandafter\def\csname PY@tok@sr\endcsname{\def\PY@tc##1{\textcolor[rgb]{0.73,0.40,0.53}{##1}}}
\expandafter\def\csname PY@tok@ss\endcsname{\def\PY@tc##1{\textcolor[rgb]{0.10,0.09,0.49}{##1}}}
\expandafter\def\csname PY@tok@sx\endcsname{\def\PY@tc##1{\textcolor[rgb]{0.00,0.50,0.00}{##1}}}
\expandafter\def\csname PY@tok@m\endcsname{\def\PY@tc##1{\textcolor[rgb]{0.40,0.40,0.40}{##1}}}
\expandafter\def\csname PY@tok@gh\endcsname{\let\PY@bf=\textbf\def\PY@tc##1{\textcolor[rgb]{0.00,0.00,0.50}{##1}}}
\expandafter\def\csname PY@tok@gu\endcsname{\let\PY@bf=\textbf\def\PY@tc##1{\textcolor[rgb]{0.50,0.00,0.50}{##1}}}
\expandafter\def\csname PY@tok@gd\endcsname{\def\PY@tc##1{\textcolor[rgb]{0.63,0.00,0.00}{##1}}}
\expandafter\def\csname PY@tok@gi\endcsname{\def\PY@tc##1{\textcolor[rgb]{0.00,0.63,0.00}{##1}}}
\expandafter\def\csname PY@tok@gr\endcsname{\def\PY@tc##1{\textcolor[rgb]{1.00,0.00,0.00}{##1}}}
\expandafter\def\csname PY@tok@ge\endcsname{\let\PY@it=\textit}
\expandafter\def\csname PY@tok@gs\endcsname{\let\PY@bf=\textbf}
\expandafter\def\csname PY@tok@gp\endcsname{\let\PY@bf=\textbf\def\PY@tc##1{\textcolor[rgb]{0.00,0.00,0.50}{##1}}}
\expandafter\def\csname PY@tok@go\endcsname{\def\PY@tc##1{\textcolor[rgb]{0.53,0.53,0.53}{##1}}}
\expandafter\def\csname PY@tok@gt\endcsname{\def\PY@tc##1{\textcolor[rgb]{0.00,0.27,0.87}{##1}}}
\expandafter\def\csname PY@tok@err\endcsname{\def\PY@bc##1{\setlength{\fboxsep}{0pt}\fcolorbox[rgb]{1.00,0.00,0.00}{1,1,1}{\strut ##1}}}
\expandafter\def\csname PY@tok@kc\endcsname{\let\PY@bf=\textbf\def\PY@tc##1{\textcolor[rgb]{0.00,0.50,0.00}{##1}}}
\expandafter\def\csname PY@tok@kd\endcsname{\let\PY@bf=\textbf\def\PY@tc##1{\textcolor[rgb]{0.00,0.50,0.00}{##1}}}
\expandafter\def\csname PY@tok@kn\endcsname{\let\PY@bf=\textbf\def\PY@tc##1{\textcolor[rgb]{0.00,0.50,0.00}{##1}}}
\expandafter\def\csname PY@tok@kr\endcsname{\let\PY@bf=\textbf\def\PY@tc##1{\textcolor[rgb]{0.00,0.50,0.00}{##1}}}
\expandafter\def\csname PY@tok@bp\endcsname{\def\PY@tc##1{\textcolor[rgb]{0.00,0.50,0.00}{##1}}}
\expandafter\def\csname PY@tok@fm\endcsname{\def\PY@tc##1{\textcolor[rgb]{0.00,0.00,1.00}{##1}}}
\expandafter\def\csname PY@tok@vc\endcsname{\def\PY@tc##1{\textcolor[rgb]{0.10,0.09,0.49}{##1}}}
\expandafter\def\csname PY@tok@vg\endcsname{\def\PY@tc##1{\textcolor[rgb]{0.10,0.09,0.49}{##1}}}
\expandafter\def\csname PY@tok@vi\endcsname{\def\PY@tc##1{\textcolor[rgb]{0.10,0.09,0.49}{##1}}}
\expandafter\def\csname PY@tok@vm\endcsname{\def\PY@tc##1{\textcolor[rgb]{0.10,0.09,0.49}{##1}}}
\expandafter\def\csname PY@tok@sa\endcsname{\def\PY@tc##1{\textcolor[rgb]{0.73,0.13,0.13}{##1}}}
\expandafter\def\csname PY@tok@sb\endcsname{\def\PY@tc##1{\textcolor[rgb]{0.73,0.13,0.13}{##1}}}
\expandafter\def\csname PY@tok@sc\endcsname{\def\PY@tc##1{\textcolor[rgb]{0.73,0.13,0.13}{##1}}}
\expandafter\def\csname PY@tok@dl\endcsname{\def\PY@tc##1{\textcolor[rgb]{0.73,0.13,0.13}{##1}}}
\expandafter\def\csname PY@tok@s2\endcsname{\def\PY@tc##1{\textcolor[rgb]{0.73,0.13,0.13}{##1}}}
\expandafter\def\csname PY@tok@sh\endcsname{\def\PY@tc##1{\textcolor[rgb]{0.73,0.13,0.13}{##1}}}
\expandafter\def\csname PY@tok@s1\endcsname{\def\PY@tc##1{\textcolor[rgb]{0.73,0.13,0.13}{##1}}}
\expandafter\def\csname PY@tok@mb\endcsname{\def\PY@tc##1{\textcolor[rgb]{0.40,0.40,0.40}{##1}}}
\expandafter\def\csname PY@tok@mf\endcsname{\def\PY@tc##1{\textcolor[rgb]{0.40,0.40,0.40}{##1}}}
\expandafter\def\csname PY@tok@mh\endcsname{\def\PY@tc##1{\textcolor[rgb]{0.40,0.40,0.40}{##1}}}
\expandafter\def\csname PY@tok@mi\endcsname{\def\PY@tc##1{\textcolor[rgb]{0.40,0.40,0.40}{##1}}}
\expandafter\def\csname PY@tok@il\endcsname{\def\PY@tc##1{\textcolor[rgb]{0.40,0.40,0.40}{##1}}}
\expandafter\def\csname PY@tok@mo\endcsname{\def\PY@tc##1{\textcolor[rgb]{0.40,0.40,0.40}{##1}}}
\expandafter\def\csname PY@tok@ch\endcsname{\let\PY@it=\textit\def\PY@tc##1{\textcolor[rgb]{0.25,0.50,0.50}{##1}}}
\expandafter\def\csname PY@tok@cm\endcsname{\let\PY@it=\textit\def\PY@tc##1{\textcolor[rgb]{0.25,0.50,0.50}{##1}}}
\expandafter\def\csname PY@tok@cpf\endcsname{\let\PY@it=\textit\def\PY@tc##1{\textcolor[rgb]{0.25,0.50,0.50}{##1}}}
\expandafter\def\csname PY@tok@c1\endcsname{\let\PY@it=\textit\def\PY@tc##1{\textcolor[rgb]{0.25,0.50,0.50}{##1}}}
\expandafter\def\csname PY@tok@cs\endcsname{\let\PY@it=\textit\def\PY@tc##1{\textcolor[rgb]{0.25,0.50,0.50}{##1}}}

\def\PYZbs{\char`\\}
\def\PYZus{\char`\_}
\def\PYZob{\char`\{}
\def\PYZcb{\char`\}}
\def\PYZca{\char`\^}
\def\PYZam{\char`\&}
\def\PYZlt{\char`\<}
\def\PYZgt{\char`\>}
\def\PYZsh{\char`\#}
\def\PYZpc{\char`\%}
\def\PYZdl{\char`\$}
\def\PYZhy{\char`\-}
\def\PYZsq{\char`\'}
\def\PYZdq{\char`\"}
\def\PYZti{\char`\~}
% for compatibility with earlier versions
\def\PYZat{@}
\def\PYZlb{[}
\def\PYZrb{]}
\makeatother


    % Exact colors from NB
    \definecolor{incolor}{rgb}{0.0, 0.0, 0.5}
    \definecolor{outcolor}{rgb}{0.545, 0.0, 0.0}



    
    % Prevent overflowing lines due to hard-to-break entities
    \sloppy 
    % Setup hyperref package
    \hypersetup{
      breaklinks=true,  % so long urls are correctly broken across lines
      colorlinks=true,
      urlcolor=urlcolor,
      linkcolor=linkcolor,
      citecolor=citecolor,
      }
    % Slightly bigger margins than the latex defaults
    
    \geometry{verbose,tmargin=1in,bmargin=1in,lmargin=1in,rmargin=1in}
    
    

    \begin{document}
    
    
    \maketitle
    
    

    
    \section{K-Means Clustering}\label{k-means-clustering}

    \subsubsection{Langkah Pertama : Pahami
dataset}\label{langkah-pertama-pahami-dataset}

    Import keperluan library untuk mengolah dan menampilkan dataset

    \begin{Verbatim}[commandchars=\\\{\}]
{\color{incolor}In [{\color{incolor}1}]:} \PY{k+kn}{import} \PY{n+nn}{pandas} \PY{k}{as} \PY{n+nn}{pd}
\end{Verbatim}


    Meload data dan mengubah data nya ke numpy array

    \begin{Verbatim}[commandchars=\\\{\}]
{\color{incolor}In [{\color{incolor}2}]:} \PY{n}{dataTrain} \PY{o}{=} \PY{n}{pd}\PY{o}{.}\PY{n}{read\PYZus{}csv}\PY{p}{(}\PY{l+s+s1}{\PYZsq{}}\PY{l+s+s1}{TrainingSet.txt}\PY{l+s+s1}{\PYZsq{}}\PY{p}{,} \PY{n}{delimiter}\PY{o}{=}\PY{l+s+s1}{\PYZsq{}}\PY{l+s+se}{\PYZbs{}t}\PY{l+s+s1}{\PYZsq{}}\PY{p}{)}\PY{o}{.}\PY{n}{as\PYZus{}matrix}\PY{p}{(}\PY{p}{)}
        \PY{n}{dataTrain}
\end{Verbatim}


\begin{Verbatim}[commandchars=\\\{\}]
{\color{outcolor}Out[{\color{outcolor}2}]:} array([[21.3 , 20.8 ],
               [20.15, 20.9 ],
               [19.2 , 21.35],
               {\ldots},
               [ 8.8 , 11.4 ],
               [ 9.  , 10.9 ],
               [ 9.35, 10.5 ]])
\end{Verbatim}
            
    Melakukan visualisasi data dengan pyplot

    \begin{Verbatim}[commandchars=\\\{\}]
{\color{incolor}In [{\color{incolor}27}]:} \PY{k+kn}{import} \PY{n+nn}{matplotlib}\PY{n+nn}{.}\PY{n+nn}{pyplot} \PY{k}{as} \PY{n+nn}{plt}
         \PY{o}{\PYZpc{}}\PY{k}{matplotlib} inline
         \PY{n}{plt}\PY{o}{.}\PY{n}{figure}\PY{p}{(}\PY{n}{figsize}\PY{o}{=}\PY{p}{(}\PY{l+m+mi}{15}\PY{p}{,} \PY{l+m+mi}{8}\PY{p}{)}\PY{p}{)}
         \PY{n}{plt}\PY{o}{.}\PY{n}{scatter}\PY{p}{(}\PY{n}{dataTrain}\PY{p}{[}\PY{p}{:}\PY{p}{,}\PY{l+m+mi}{0}\PY{p}{]}\PY{p}{,}\PY{n}{dataTrain}\PY{p}{[}\PY{p}{:}\PY{p}{,}\PY{l+m+mi}{1}\PY{p}{]}\PY{p}{)} \PY{c+c1}{\PYZsh{} dataTrain[:,0] mengambil kolom X, sedangkan [:,1] mengambil kolom Y}
         \PY{n}{plt}\PY{o}{.}\PY{n}{title}\PY{p}{(}\PY{l+s+s1}{\PYZsq{}}\PY{l+s+s1}{Scatter Plot}\PY{l+s+s1}{\PYZsq{}}\PY{p}{)}
         \PY{n}{plt}\PY{o}{.}\PY{n}{xlabel}\PY{p}{(}\PY{l+s+s1}{\PYZsq{}}\PY{l+s+s1}{x}\PY{l+s+s1}{\PYZsq{}}\PY{p}{)}
         \PY{n}{plt}\PY{o}{.}\PY{n}{ylabel}\PY{p}{(}\PY{l+s+s1}{\PYZsq{}}\PY{l+s+s1}{y}\PY{l+s+s1}{\PYZsq{}}\PY{p}{)}
\end{Verbatim}


\begin{Verbatim}[commandchars=\\\{\}]
{\color{outcolor}Out[{\color{outcolor}27}]:} Text(0,0.5,'y')
\end{Verbatim}
            
    \begin{center}
    \adjustimage{max size={0.9\linewidth}{0.9\paperheight}}{output_7_1.png}
    \end{center}
    { \hspace*{\fill} \\}
    
    Hasil datasetnya adalah seperti di tabel scatter plot diatas, sekilas
terlihat terdapat 7 cluster data

    \subsection{Menentukan jumlah data}\label{menentukan-jumlah-data}

    Menggunakan Library Sci-Kit Learn untuk membantu menentukan jumlah
cluster

    \begin{Verbatim}[commandchars=\\\{\}]
{\color{incolor}In [{\color{incolor}22}]:} \PY{k+kn}{from} \PY{n+nn}{sklearn}\PY{n+nn}{.}\PY{n+nn}{cluster} \PY{k}{import} \PY{n}{KMeans}
         \PY{n}{sse} \PY{o}{=} \PY{p}{[}\PY{p}{]}
         \PY{k}{for} \PY{n}{i} \PY{o+ow}{in} \PY{n+nb}{range}\PY{p}{(}\PY{l+m+mi}{1}\PY{p}{,}\PY{l+m+mi}{10}\PY{p}{)}\PY{p}{:}
             \PY{n}{kmeans} \PY{o}{=} \PY{n}{KMeans}\PY{p}{(}\PY{n}{n\PYZus{}clusters} \PY{o}{=} \PY{n}{i}\PY{p}{,} \PY{n}{init} \PY{o}{=}\PY{l+s+s1}{\PYZsq{}}\PY{l+s+s1}{k\PYZhy{}means++}\PY{l+s+s1}{\PYZsq{}}\PY{p}{,} \PY{n}{max\PYZus{}iter} \PY{o}{=} \PY{l+m+mi}{300}\PY{p}{,} \PY{n}{n\PYZus{}init} \PY{o}{=} \PY{l+m+mi}{10}\PY{p}{,} \PY{n}{random\PYZus{}state} \PY{o}{=}\PY{l+m+mi}{0}\PY{p}{)}
             \PY{n}{kmeans}\PY{o}{.}\PY{n}{fit}\PY{p}{(}\PY{n}{dataTrain}\PY{p}{)}
             \PY{n}{sse}\PY{o}{.}\PY{n}{append}\PY{p}{(}\PY{n}{kmeans}\PY{o}{.}\PY{n}{inertia\PYZus{}}\PY{p}{)}
         \PY{n}{plt}\PY{o}{.}\PY{n}{figure}\PY{p}{(}\PY{n}{figsize}\PY{o}{=}\PY{p}{(}\PY{l+m+mi}{15}\PY{p}{,}\PY{l+m+mi}{8}\PY{p}{)}\PY{p}{)}
         \PY{n}{plt}\PY{o}{.}\PY{n}{plot}\PY{p}{(}\PY{n+nb}{range}\PY{p}{(}\PY{l+m+mi}{1}\PY{p}{,}\PY{l+m+mi}{10}\PY{p}{)}\PY{p}{,} \PY{n}{sse}\PY{p}{)}
         \PY{n}{plt}\PY{o}{.}\PY{n}{title}\PY{p}{(}\PY{l+s+s1}{\PYZsq{}}\PY{l+s+s1}{Elbow Method}\PY{l+s+s1}{\PYZsq{}}\PY{p}{)}
         \PY{n}{plt}\PY{o}{.}\PY{n}{xlabel}\PY{p}{(}\PY{l+s+s1}{\PYZsq{}}\PY{l+s+s1}{Jumlah Cluster}\PY{l+s+s1}{\PYZsq{}}\PY{p}{)}
         \PY{n}{plt}\PY{o}{.}\PY{n}{ylabel}\PY{p}{(}\PY{l+s+s1}{\PYZsq{}}\PY{l+s+s1}{SSE}\PY{l+s+s1}{\PYZsq{}}\PY{p}{)}
\end{Verbatim}


\begin{Verbatim}[commandchars=\\\{\}]
{\color{outcolor}Out[{\color{outcolor}22}]:} Text(0,0.5,'SSE')
\end{Verbatim}
            
    \begin{center}
    \adjustimage{max size={0.9\linewidth}{0.9\paperheight}}{output_11_1.png}
    \end{center}
    { \hspace*{\fill} \\}
    
    \subsection{Membangun K-Means tanpa
library}\label{membangun-k-means-tanpa-library}

    \begin{Verbatim}[commandchars=\\\{\}]
{\color{incolor}In [{\color{incolor}5}]:} \PY{c+c1}{\PYZsh{} Mengimport library library yang diperlukan}
        \PY{k+kn}{import} \PY{n+nn}{numpy} \PY{k}{as} \PY{n+nn}{np}
        \PY{k+kn}{from} \PY{n+nn}{collections} \PY{k}{import} \PY{n}{defaultdict}
        \PY{k+kn}{from} \PY{n+nn}{random} \PY{k}{import} \PY{n}{randint} \PY{k}{as} \PY{n}{rand}
        
        \PY{c+c1}{\PYZsh{} Pengimplementasian K\PYZhy{}Means Algorithm dengan melakukan pendekatan OOP. }
        \PY{k}{class} \PY{n+nc}{K\PYZus{}Means}\PY{p}{(}\PY{n+nb}{object}\PY{p}{)}\PY{p}{:}
            \PY{k}{def} \PY{n+nf}{\PYZus{}\PYZus{}init\PYZus{}\PYZus{}}\PY{p}{(}\PY{n+nb+bp}{self}\PY{p}{,} \PY{n}{K}\PY{p}{)}\PY{p}{:}
                \PY{n+nb+bp}{self}\PY{o}{.}\PY{n}{K} \PY{o}{=} \PY{n}{K}
            
            \PY{n+nd}{@staticmethod}
            \PY{k}{def} \PY{n+nf}{new\PYZus{}centroids}\PY{p}{(}\PY{n}{clusters}\PY{p}{)}\PY{p}{:}
                \PY{l+s+sd}{\PYZdq{}\PYZdq{}\PYZdq{}}
        \PY{l+s+sd}{        Fungsi ini akan mengembalikan centroid yang terbaru setelah menghitung}
        \PY{l+s+sd}{        rata\PYZhy{}rata dari setiap cluster}
        \PY{l+s+sd}{        \PYZdq{}\PYZdq{}\PYZdq{}}
                \PY{k}{return} \PY{p}{[}\PY{n}{np}\PY{o}{.}\PY{n}{mean}\PY{p}{(}\PY{n}{i}\PY{p}{,} \PY{n}{axis}\PY{o}{=}\PY{l+m+mi}{0}\PY{p}{)} \PY{k}{for} \PY{n}{i} \PY{o+ow}{in} \PY{n}{clusters}\PY{o}{.}\PY{n}{values}\PY{p}{(}\PY{p}{)}\PY{p}{]}
            
            \PY{n+nd}{@classmethod}
            \PY{k}{def} \PY{n+nf}{clustering}\PY{p}{(}\PY{n+nb+bp}{cls}\PY{p}{,} \PY{n}{clusters}\PY{p}{,} \PY{n}{data}\PY{p}{,} \PY{n}{rd\PYZus{}points}\PY{p}{)}\PY{p}{:}
                \PY{l+s+sd}{\PYZdq{}\PYZdq{}\PYZdq{}}
        \PY{l+s+sd}{        Fungsi ini akan memasukan data kedalam kelompok cluster masing\PYZhy{}masing}
        \PY{l+s+sd}{        \PYZdq{}\PYZdq{}\PYZdq{}}
                \PY{k}{for} \PY{n}{i} \PY{o+ow}{in} \PY{n}{data}\PY{p}{:}
                    \PY{n}{clusters}\PY{p}{[}\PY{n+nb+bp}{cls}\PY{o}{.}\PY{n}{lable}\PY{p}{(}\PY{n}{data\PYZus{}point}\PY{o}{=}\PY{n}{i}\PY{p}{,} \PY{n}{points}\PY{o}{=}\PY{n}{rd\PYZus{}points}\PY{p}{)}\PY{p}{]}\PY{o}{.}\PY{n}{append}\PY{p}{(}\PY{n}{i}\PY{p}{)}
                \PY{k}{return}
            
            \PY{n+nd}{@classmethod} 
            \PY{k}{def} \PY{n+nf}{lable}\PY{p}{(}\PY{n+nb+bp}{cls}\PY{p}{,} \PY{n}{data\PYZus{}point}\PY{p}{,} \PY{n}{points}\PY{p}{)}\PY{p}{:}
                \PY{l+s+sd}{\PYZdq{}\PYZdq{}\PYZdq{}}
        \PY{l+s+sd}{        Fungsi ini akan mengembalikan label dari data point}
        \PY{l+s+sd}{        \PYZdq{}\PYZdq{}\PYZdq{}}
                \PY{n}{values} \PY{o}{=} \PY{p}{[}\PY{n+nb+bp}{cls}\PY{o}{.}\PY{n}{euclid\PYZus{}dis}\PY{p}{(}\PY{n}{data\PYZus{}point}\PY{p}{,} \PY{n}{i}\PY{p}{)} \PY{k}{for} \PY{n}{i} \PY{o+ow}{in} \PY{n}{points}\PY{p}{]}
                \PY{k}{return} \PY{n+nb}{min}\PY{p}{(}\PY{n+nb}{range}\PY{p}{(}\PY{n+nb}{len}\PY{p}{(}\PY{n}{values}\PY{p}{)}\PY{p}{)}\PY{p}{,} \PY{n}{key}\PY{o}{=}\PY{n}{values}\PY{o}{.}\PY{n+nf+fm}{\PYZus{}\PYZus{}getitem\PYZus{}\PYZus{}}\PY{p}{)}
            
            \PY{n+nd}{@classmethod}
            \PY{k}{def} \PY{n+nf}{centroids}\PY{p}{(}\PY{n+nb+bp}{cls}\PY{p}{,} \PY{n}{data}\PY{p}{,} \PY{n}{K}\PY{p}{)}\PY{p}{:}
                \PY{l+s+sd}{\PYZdq{}\PYZdq{}\PYZdq{}}
        \PY{l+s+sd}{        Memilih centroid sebanyak K dengan memilih nya dari index}
        \PY{l+s+sd}{        dan dijadikan sebagai centroid}
        \PY{l+s+sd}{        \PYZdq{}\PYZdq{}\PYZdq{}}
                \PY{n}{mean\PYZus{}p} \PY{o}{=} \PY{n}{np}\PY{o}{.}\PY{n}{mean}\PY{p}{(}\PY{n}{data}\PY{p}{,} \PY{n}{axis}\PY{o}{=}\PY{l+m+mi}{0}\PY{p}{)}
                \PY{n}{dist\PYZus{}log} \PY{o}{=} \PY{p}{[}\PY{p}{]}
                \PY{n}{result} \PY{o}{=} \PY{p}{[}\PY{p}{]}
                \PY{k}{while} \PY{n+nb}{len}\PY{p}{(}\PY{n}{result}\PY{p}{)} \PY{o}{\PYZlt{}} \PY{n}{K}\PY{p}{:}
                    \PY{k}{if} \PY{o+ow}{not} \PY{n}{dist\PYZus{}log}\PY{p}{:}
                        \PY{n}{p} \PY{o}{=} \PY{n}{data}\PY{p}{[}\PY{n}{rand}\PY{p}{(}\PY{l+m+mi}{0}\PY{p}{,} \PY{n}{K} \PY{o}{\PYZhy{}} \PY{l+m+mi}{1}\PY{p}{)}\PY{p}{]}
                        \PY{n}{result}\PY{o}{.}\PY{n}{insert}\PY{p}{(}\PY{l+m+mi}{0}\PY{p}{,} \PY{n}{p}\PY{p}{)}
                        \PY{n}{dist\PYZus{}log}\PY{o}{.}\PY{n}{insert}\PY{p}{(}\PY{l+m+mi}{0}\PY{p}{,} \PY{n+nb+bp}{cls}\PY{o}{.}\PY{n}{euclid\PYZus{}dis}\PY{p}{(}\PY{n}{mean\PYZus{}p}\PY{p}{,} \PY{n}{p}\PY{p}{)}\PY{p}{)}
                    \PY{k}{else}\PY{p}{:}
                        \PY{k}{for} \PY{n}{i} \PY{o+ow}{in} \PY{n}{data}\PY{p}{:}
                            \PY{n}{p} \PY{o}{=} \PY{n}{i}
                            \PY{k}{if} \PY{n}{np}\PY{o}{.}\PY{n}{any}\PY{p}{(}\PY{n}{result} \PY{o}{==} \PY{n}{p}\PY{p}{)}\PY{p}{:}
                                \PY{k}{continue}
                            \PY{n}{d} \PY{o}{=} \PY{n+nb+bp}{cls}\PY{o}{.}\PY{n}{euclid\PYZus{}dis}\PY{p}{(}\PY{n}{p}\PY{p}{,} \PY{n}{result}\PY{p}{[}\PY{l+m+mi}{0}\PY{p}{]}\PY{p}{)}
                            \PY{k}{if} \PY{n}{d} \PY{o}{\PYZgt{}} \PY{n}{dist\PYZus{}log}\PY{p}{[}\PY{l+m+mi}{0}\PY{p}{]}\PY{p}{:}
                                \PY{n}{result}\PY{o}{.}\PY{n}{insert}\PY{p}{(}\PY{l+m+mi}{0}\PY{p}{,} \PY{n}{p}\PY{p}{)}
                                \PY{n}{dist\PYZus{}log}\PY{o}{.}\PY{n}{insert}\PY{p}{(}\PY{l+m+mi}{0}\PY{p}{,} \PY{n+nb+bp}{cls}\PY{o}{.}\PY{n}{euclid\PYZus{}dis}\PY{p}{(}\PY{n}{mean\PYZus{}p}\PY{p}{,} \PY{n}{p}\PY{p}{)}\PY{p}{)}
                                \PY{k}{break}
                \PY{k}{return} \PY{n}{result}    
                
            \PY{n+nd}{@staticmethod}
            \PY{k}{def} \PY{n+nf}{euclid\PYZus{}dis}\PY{p}{(}\PY{n}{a}\PY{p}{,} \PY{n}{b}\PY{p}{)}\PY{p}{:}
                \PY{l+s+sd}{\PYZdq{}\PYZdq{}\PYZdq{}}
        \PY{l+s+sd}{        Menghitung jarak dengan fungsi norm dari numpy untuk menghitung euclidean distance}
        \PY{l+s+sd}{        \PYZdq{}\PYZdq{}\PYZdq{}}
                \PY{k}{return} \PY{n}{np}\PY{o}{.}\PY{n}{linalg}\PY{o}{.}\PY{n}{norm}\PY{p}{(}\PY{n}{a} \PY{o}{\PYZhy{}} \PY{n}{b}\PY{p}{)}
                
            \PY{k}{def} \PY{n+nf}{fit}\PY{p}{(}\PY{n+nb+bp}{self}\PY{p}{,} \PY{n}{data}\PY{p}{)}\PY{p}{:}
                \PY{l+s+sd}{\PYZdq{}\PYZdq{}\PYZdq{}}
        \PY{l+s+sd}{        Fungsi utama yang akan mengembalikan label data}
        \PY{l+s+sd}{        \PYZdq{}\PYZdq{}\PYZdq{}}
                \PY{c+c1}{\PYZsh{} initialize centroids randomly}
                \PY{n}{rd\PYZus{}points} \PY{o}{=} \PY{n}{K\PYZus{}Means}\PY{o}{.}\PY{n}{centroids}\PY{p}{(}\PY{n}{data}\PY{p}{,} \PY{n}{K}\PY{o}{=}\PY{n+nb+bp}{self}\PY{o}{.}\PY{n}{K}\PY{p}{)}
                
                \PY{c+c1}{\PYZsh{} initialize book keeping variables}
                \PY{n}{counter} \PY{o}{=} \PY{k+kc}{False}
                \PY{n}{clusters} \PY{o}{=} \PY{k+kc}{None}
                \PY{n}{change\PYZus{}log} \PY{o}{=} \PY{p}{[}\PY{p}{]}
                
                \PY{c+c1}{\PYZsh{} initializing the main loop}
                \PY{k}{while} \PY{n}{counter} \PY{o+ow}{is} \PY{k+kc}{False}\PY{p}{:}
                    \PY{c+c1}{\PYZsh{} we will store each cluster in a list}
                    \PY{n}{clusters} \PY{o}{=} \PY{n}{defaultdict}\PY{p}{(}\PY{n+nb}{list}\PY{p}{)}
                    
                    \PY{c+c1}{\PYZsh{} putting the data into their own clusters based on the centroids}
                    \PY{n}{K\PYZus{}Means}\PY{o}{.}\PY{n}{clustering}\PY{p}{(}\PY{n}{clusters}\PY{p}{,} \PY{n}{data}\PY{p}{,} \PY{n}{rd\PYZus{}points}\PY{p}{)}
                     
                    \PY{c+c1}{\PYZsh{} change centroid based on cluster mean}
                    \PY{n}{rd\PYZus{}points} \PY{o}{=} \PY{n}{K\PYZus{}Means}\PY{o}{.}\PY{n}{new\PYZus{}centroids}\PY{p}{(}\PY{n}{clusters}\PY{p}{)}
                    
                    \PY{c+c1}{\PYZsh{} now we are going to see if there are any changes in the cluster}
                    \PY{n}{temp} \PY{o}{=} \PY{p}{[}\PY{n+nb}{len}\PY{p}{(}\PY{n}{i}\PY{p}{)} \PY{k}{for} \PY{n}{i} \PY{o+ow}{in} \PY{n}{clusters}\PY{o}{.}\PY{n}{values}\PY{p}{(}\PY{p}{)}\PY{p}{]}
                    \PY{k}{if} \PY{o+ow}{not} \PY{n}{change\PYZus{}log}\PY{p}{:}
                        \PY{n}{change\PYZus{}log} \PY{o}{=} \PY{n}{temp}
                    \PY{k}{else}\PY{p}{:}
                        \PY{k}{if} \PY{n}{temp} \PY{o}{==} \PY{n}{change\PYZus{}log}\PY{p}{:}
                            \PY{n}{counter} \PY{o}{=} \PY{k+kc}{True}
                        \PY{k}{else}\PY{p}{:}
                            \PY{n}{change\PYZus{}log} \PY{o}{=} \PY{n}{temp}
                            
                \PY{n+nb+bp}{self}\PY{o}{.}\PY{n}{points} \PY{o}{=} \PY{n}{rd\PYZus{}points}
                \PY{n+nb+bp}{self}\PY{o}{.}\PY{n}{data} \PY{o}{=} \PY{n}{data}
                
            \PY{k}{def} \PY{n+nf}{labels}\PY{p}{(}\PY{n+nb+bp}{self}\PY{p}{)}\PY{p}{:}
                \PY{l+s+sd}{\PYZdq{}\PYZdq{}\PYZdq{}}
        \PY{l+s+sd}{        Fungsi ini akan mengembalikan label dari setiap data yang ada di dataset}
        \PY{l+s+sd}{        \PYZdq{}\PYZdq{}\PYZdq{}}
                \PY{n}{r} \PY{o}{=} \PY{p}{[}\PY{p}{]}
                \PY{k}{for} \PY{n}{i} \PY{o+ow}{in} \PY{n+nb+bp}{self}\PY{o}{.}\PY{n}{data}\PY{p}{:}
                    \PY{n}{r}\PY{o}{.}\PY{n}{append}\PY{p}{(}\PY{n}{K\PYZus{}Means}\PY{o}{.}\PY{n}{lable}\PY{p}{(}\PY{n}{i}\PY{p}{,} \PY{n+nb+bp}{self}\PY{o}{.}\PY{n}{points}\PY{p}{)}\PY{p}{)}
                \PY{k}{return} \PY{n}{np}\PY{o}{.}\PY{n}{array}\PY{p}{(}\PY{n}{r}\PY{p}{)}
                
\end{Verbatim}


    \begin{Verbatim}[commandchars=\\\{\}]
{\color{incolor}In [{\color{incolor}6}]:} \PY{k}{def} \PY{n+nf}{warna}\PY{p}{(}\PY{n}{x}\PY{p}{)}\PY{p}{:}
            \PY{l+s+sd}{\PYZdq{}\PYZdq{}\PYZdq{}}
        \PY{l+s+sd}{    Fungsi ini akan mengembalikan char kode warna untuk setiap input integer}
        \PY{l+s+sd}{    \PYZdq{}\PYZdq{}\PYZdq{}}
            \PY{k}{return} \PY{p}{\PYZob{}}
                \PY{l+m+mi}{0} \PY{p}{:} \PY{l+s+s1}{\PYZsq{}}\PY{l+s+s1}{r}\PY{l+s+s1}{\PYZsq{}}\PY{p}{,}
                \PY{l+m+mi}{1} \PY{p}{:} \PY{l+s+s1}{\PYZsq{}}\PY{l+s+s1}{g}\PY{l+s+s1}{\PYZsq{}}\PY{p}{,}
                \PY{l+m+mi}{2} \PY{p}{:} \PY{l+s+s1}{\PYZsq{}}\PY{l+s+s1}{b}\PY{l+s+s1}{\PYZsq{}}\PY{p}{,}
                \PY{l+m+mi}{3} \PY{p}{:} \PY{l+s+s1}{\PYZsq{}}\PY{l+s+s1}{c}\PY{l+s+s1}{\PYZsq{}}\PY{p}{,}
                \PY{l+m+mi}{4} \PY{p}{:} \PY{l+s+s1}{\PYZsq{}}\PY{l+s+s1}{y}\PY{l+s+s1}{\PYZsq{}}\PY{p}{,}
                \PY{l+m+mi}{5} \PY{p}{:} \PY{l+s+s1}{\PYZsq{}}\PY{l+s+s1}{m}\PY{l+s+s1}{\PYZsq{}}\PY{p}{,}
                \PY{l+m+mi}{6} \PY{p}{:} \PY{l+s+s1}{\PYZsq{}}\PY{l+s+s1}{k}\PY{l+s+s1}{\PYZsq{}}\PY{p}{,}
            \PY{p}{\PYZcb{}}\PY{p}{[}\PY{n}{x}\PY{p}{]}
\end{Verbatim}


    \subsection{Melakukan plotting data train dengan K =
3}\label{melakukan-plotting-data-train-dengan-k-3}

    \begin{Verbatim}[commandchars=\\\{\}]
{\color{incolor}In [{\color{incolor}38}]:} \PY{c+c1}{\PYZsh{} Membuat objek model K\PYZhy{}Means dengan K = 3}
         \PY{n}{model} \PY{o}{=} \PY{n}{K\PYZus{}Means}\PY{p}{(}\PY{l+m+mi}{3}\PY{p}{)}
         \PY{c+c1}{\PYZsh{} Melakukan pembuatan model}
         \PY{n}{model}\PY{o}{.}\PY{n}{fit}\PY{p}{(}\PY{n}{dataTrain}\PY{p}{)}
         \PY{c+c1}{\PYZsh{} Menampung label hasil training}
         \PY{n}{labels} \PY{o}{=} \PY{n}{model}\PY{o}{.}\PY{n}{labels}\PY{p}{(}\PY{p}{)}
         \PY{c+c1}{\PYZsh{} Mengubah label yang terdiri dari integer menjadi char r/g/b/c/m/y/k}
         \PY{n}{labelColor} \PY{o}{=} \PY{p}{[}\PY{p}{]}
         \PY{k}{for} \PY{n}{l} \PY{o+ow}{in} \PY{n}{labels}\PY{p}{:}
             \PY{n}{labelColor}\PY{o}{.}\PY{n}{append}\PY{p}{(}\PY{n}{warna}\PY{p}{(}\PY{n}{l}\PY{p}{)}\PY{p}{)}
         
         \PY{c+c1}{\PYZsh{} Memplot hasil}
         \PY{n}{plt}\PY{o}{.}\PY{n}{figure}\PY{p}{(}\PY{n}{figsize}\PY{o}{=}\PY{p}{(}\PY{l+m+mi}{15}\PY{p}{,} \PY{l+m+mi}{8}\PY{p}{)}\PY{p}{)}
         \PY{n}{plt}\PY{o}{.}\PY{n}{scatter}\PY{p}{(}\PY{n}{dataTrain}\PY{p}{[}\PY{p}{:}\PY{p}{,}\PY{l+m+mi}{0}\PY{p}{]}\PY{p}{,} \PY{n}{dataTrain}\PY{p}{[}\PY{p}{:}\PY{p}{,}\PY{l+m+mi}{1}\PY{p}{]}\PY{p}{,} \PY{n}{c}\PY{o}{=}\PY{n}{labelColor}\PY{p}{)}
         \PY{n}{plt}\PY{o}{.}\PY{n}{show}\PY{p}{(}\PY{p}{)}
\end{Verbatim}


    \begin{center}
    \adjustimage{max size={0.9\linewidth}{0.9\paperheight}}{output_16_0.png}
    \end{center}
    { \hspace*{\fill} \\}
    
    \begin{Verbatim}[commandchars=\\\{\}]
{\color{incolor}In [{\color{incolor} }]:} \PY{n}{Mencob}
\end{Verbatim}



    % Add a bibliography block to the postdoc
    
    
    
    \end{document}
